
%%%-----------------------------------------------------------%
%% Cambia colori da thema default
%% Questi sono i due colori ufficiali rosso e grigio
\definecolor {cfred}{rgb}{0.709,0.196,0.329} 	%{ 181 ,50 ,84}
\definecolor {cfgrey}{rgb}{0.537,0.537,0.537} 	%{ 137,137,137}
\definecolor {cflink}{rgb}{0.615,0.615,0.607} 	%{157,157,155}

\setbeamercolor{palette primary}{bg=cfred,fg=white}
\setbeamercolor{palette secondary}{bg=cfred,fg=white}
\setbeamercolor{palette tertiary}{bg=cfred,fg=white}
\setbeamercolor{palette quaternary}{bg=cfred,fg=white}
\setbeamercolor{structure}{fg=cfred}		 % itemize, enumerate, etc
\setbeamercolor{section in toc}{fg=cfred} 		 % TOC sections
% Override palette coloring with secondary
\setbeamercolor{subsection in head/foot}{bg=cfgrey,fg=white}
%%%------------------------------------------------------------

%% Definisce il blocco con riquadro che non è presente nel tema default (commentare se si usano altri temi)
\setbeamercolor{uppercolor}{fg=white,bg=cfred}%
\setbeamercolor{lowercolor}{fg=black,bg=white}%
\def \bblock{\begin{beamerboxesrounded}[upper=uppercolor,lower=lowercolor,shadow=true]}
\def \eblock{\end{beamerboxesrounded}}
%%-----------------------------------------------------------

%% Intestazione ripetuta per ogni slide
\addtobeamertemplate{headline}{%
\vspace{0.25cm} \ \ 
\includegraphics[height=1.0cm]{logobeamEN.png} 	% sostituire con logobeamIT.png per italiano
\hspace{0.641\textwidth}{\color{cflink} {\small www.unive.it}} %per 16:9
%\hspace{0.551\textwidth}{\color{cflink} {\small www.unive.it}} %per 4::3

\vspace{0.25cm}
{\color{cfred} \hrule \hrule  }
\textbf{}
}{}
%%-------------------------------------------------------------


%------------------------------------------------------------
%The next block of commands puts the table of contents at the 
%beginning of each section and highlights the current section:

\AtBeginSection[]
{
  \begin{frame}
    \frametitle{Table of Contents}
    \tableofcontents[currentsection]
  \end{frame}
}
%------------------------------------------------------------

% poniższy złoty kawałek kodu przy rozpoczęciu każdej nowej sekcji robi slajd pokazujący, gdzie jesteśmy
% \AtBeginSection[]{
%   \begin{frame}{Outline}
%   \setbeamertemplate{section in toc}[sections numbered]
%   \tableofcontents[currentsection, hideothersubsections]
%   \end{frame} 
% }

% \AtBeginSubsection[]{
%   \begin{frame}{Outline}
%   \setbeamertemplate{section in toc}[sections numbered]
%   \tableofcontents[currentsection, currentsubsection]
%   \end{frame} 
% }


% grey text
\definecolor{lightgray}{RGB}{211,211,211}
\newcommand\gray[1]{{\color{lightgray}{#1}}}

% compatibility colours
\definecolor{cinnamon}{RGB}{205,92,92}

% emoji for SHAP working example
\usepackage{fontspec}  % emoji
\newfontfamily{\NotoEmoji} {NotoColorEmoji.ttf}[Renderer=Harfbuzz]
% % % % % % % % % % % % % % % % % % %
%   THE COMPILER MUST BE LUALATEX   %
% % % % % % % % % % % % % % % % % % %

